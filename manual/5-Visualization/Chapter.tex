\chapter{Visualization}\label{chap:visualization}

\vampire provides tools for visualising systems using external programs such as Rasmol, Jmol and POV-Ray. To compile these utilities, use the following command in the main directory of your \vampire installation folder:

\begin{minipage}[c]{\textwidth}
\centering
\textit{make vdc}
\end{minipage}\\

 The \vampire data converter, or vdc, is run to produce the input files needed. By default, it provides output for both Rasmol and POV-Ray.

%opt/vampire/cfg2xx\\

\section*{Getting started}
\addcontentsline{toc}{section}{Getting started}
To generate the positions of your atoms, multiple parameters must be set in the input file, depending on the number of cores the simulation is run on.
min/max\\

\subsection*{input}
{\footnotesize
\begin{verbatim}
#------------------------------------------
# data output
#------------------------------------------
config:atoms
config:output-nodes = 12
config:output-rate     = 1000
config:output-format = binary
\end{verbatim}
}

add descriptors for each parameter\\

\section*{Atomic visualization with rasmol}
\addcontentsline{toc}{section}{Atomic visualization with rasmol}

To visualise your system using Rasmol, simply run vdc in the same directory as your output. The config:output files must be present.\\

This produced a file called crystal.xyz, which is a chemical file format with information on the atomic positions. The format of the .xyz format is as follows:\\

\subsection*{.xyz}
{\footnotesize
\begin{verbatim}
<number of atoms>
comment line
<element> <X> <Y> <Z>
...
\end{verbatim}
}

The element in the .xyz file does not necessarily need to be the same as the atoms used in your system. They can instead be chosen for a different colour palette depending on the users requirements.

\section*{Atomic visualization with POV-Ray}
\addcontentsline{toc}{section}{Atomic visualization with POV-Ray}

To produce pictures of your material of punishable quality and high configurability, it is also possible to use POV-Ray. After running vdc, the file "spins.pov" contains all the necessary information and an image may be produced by using:

\begin{minipage}[c]{\textwidth}
\centering
\textit{povray spins.pov}
\end{minipage}\\

Depending on the parameters included in the input file, several snapshots of the system may be produced and using the above command will render images for each snapshot. To select specific snapshots or ranges, you need to add the following flags:

\begin{minipage}[c]{\textwidth}
\centering
\textit{+KFF<N> (initial frame number)\\
+KFI<N> (final frame number)}
\end{minipage}\\

For example, to render frame 9 only, you could use:

\begin{minipage}[c]{\textwidth}
\centering
\textit{povray -W3600 -H2700 +A0.3 +KFI9 +KFF9 spins.pov}
\end{minipage}\\

Where the "-W" and "-H" flags define the width and heigh of the image (the resolution), and "+A" is used for antialiasing. \\

\section{Customisation options}
The POV-Ray output from vdc can be customised in several ways by using command line flags when running vdc. There are several choices of possible colourmap configurations, the ones provided by default are made to be perceptually uniform and in some cases take account of colourblindness. Information on the colourmaps, the importance of perceptually uniform maps and how to adapt and use different maps can be found from "Peter Kovesi. Good Colour Maps: How to Design Them. arXiv:1509.03700 [cs.GR] 2015". \\

\subsection{Colourmaps}

\begin{minipage}[c]{\textwidth}
\centering
\textit{-{}-colourmap [CBWR - default/C2/BWR/Rainbow]}
\end{minipage}\\

\begin{figure*}[!h]
\center
\includegraphics[width=4cm]{figures/CBWR_colourmap.png}
%\caption{}
\label{fig:CBWR_colourmap}
\end{figure*}

By default, a 1D colourmap is used. Aligned along the z-axis, spins in the \{0,0,1\} direction are red, while spins antiparallel to this \{0,0,-1\} are blue. Between these values, the colour transitions to white around the x-y plane. This corresponds to the "CBWR" colourmap, a cyclic blue-white-red map, which lends itself well to 1D or 2D spin sytems where there are two principle spin directions, such as antiferromagnets and ferrimagnets. Some care must be taken to align the principle spin directions with the z-axis, as this is the axis along which colour is applied. This can also be changed using the "-{}-vector-z" command line argument.   \\

\begin{figure*}[!h]
\center
\includegraphics[width=4cm]{figures/C2_colourmap.png}
%\caption{}
\label{fig:C2_colourmap}
\end{figure*}

The "C2" coloumap is also cyclic and useful for 3D magnetic systems such as vortex states. It has four principle directions of magenta, yellow, green and blue. As it is cyclic, there will be a smooth transition between colour at all angles, irrespective of what is chosen as the zero degree spin direction. Sytems which benefit from this colourmap may also use the "-{}-3D" command line argument which applies a darkening and britening effect along the x-axis. \\

\begin{figure*}[!h]
\center
\includegraphics[width=7cm]{figures/BWR_colourmap.png}
%\caption{}
\label{fig:C2_colourmap}
\end{figure*}

The BWR colourmap is very similar in properties to the CBWR map however it is not cyclic. This mean that spins along the positive z-axis will be red with a small positive y-component and blue with a small negative y-component. There will be an immediate flip from bright red to blue as this transition occurs. This can be used to emphasise the transition between spin directions. The transition point can be changed by using the "-{}-vector-z" command line argument. \\

\begin{figure*}[!h]
\center
\includegraphics[width=7cm]{figures/Rainbow_colourmap.png}
%\caption{}
\label{fig:Rainbow_colourmap}
\end{figure*}

The "Rainbow" colourmap can be used in 2D systems where spins are aligned in many different directions such as high temperature simulations. While it is still designed to be somewhat perceptually uniform, this is very difficult to do with rainbow palettes hence its use typically loses detail when compared to other maps, however it is also one of the most vibrant. \\

\begin{minipage}[c]{\textwidth}
\centering
\textit{vdc spins.pov -{}-custom-colourmap file-name}
\end{minipage}\\

\subsection{Custom Colourmaps}

A user defined colourmap can also be used. To apply a different map, a file containing 256 colours in the RBG format must be provided in the same directory that vdc is run. RGB values must be space separated, with no other information such as line numbers. The beginning of an example colourmap is shown below. Pregenerated perceptually uniform colourmaps of various forms, including those included in vampire by default, can be found in "peterkovesi.com/projects/colourmaps/index.html" under the Download secion. \\

\subsection*{custom\_colourmap\_file}
{\footnotesize
\begin{verbatim}
0.000000 0.000000 0.000000
0.005561 0.005563 0.005563
0.011212 0.011219 0.011217
0.016877 0.016885 0.016883
0.022438 0.022448 0.022445
0.027998 0.028011 0.028008
0.033540 0.033554 0.033551
0.039316 0.039333 0.039329
0.044700 0.044719 0.044714
0.049695 0.049713 0.049709
0.054322 0.054343 0.054338
\end{verbatim}
}

\subsection{3D Systems}

POV-Ray images produced by vdc can have a 3D brightening effect applied by using the "-{}-3D" command line argument. When spins do not lie only in the yz-plane, their colour brightness is increased or reduced depending on their magnitude in the x-axis.

\begin{minipage}[c]{\textwidth}
\centering
\textit{vdc spins.pov -{}-3D}
\end{minipage}\\

\subsection{Defining axes}

\begin{minipage}[c]{\textwidth}
\centering
\textit{vdc spins.pov -{}-vector-z \textbackslash\{x,y,z\textbackslash\}}
\end{minipage}\\

To define the axis, simply use the command "-{}-vector-z" followed by a direction vector. This does not need to be normalised, and can be used without defining the xy-plane, as explained below.\\

Similarly, another direction can be defined for the $0^{\circ}$ direction in the xy-plane. However, as the z-axis effect is not necessarily in the \{0,0,1\} direction, this parameter cannot be used alone and must be accompanied by the flag "-{}-vector-z".

\begin{minipage}[c]{\textwidth}
\centering
\textit{vdc spins.pov -{}-vector-z \textbackslash\{x,y,z\textbackslash\} -{}-vector-x \textbackslash\{x,y,z\textbackslash\}}
\end{minipage}\\

\section*{Micromagnetic visualization with PovRAY}
\addcontentsline{toc}{section}{Micromagnetic visualization with PovRAY}
cell2povray\\
macros\\
customization\\
colouring options\\

\section*{Visualization Movies}
\addcontentsline{toc}{section}{Visualization Movies}

\chapter{Visualization}\label{chap:visualization}

\vampire provides tools for visualising systems using external programs such as
Rasmol, Jmol and POV-Ray. To compile these utilities, use the following command
in the main directory of your \vampire installation folder:

\noindent
\begin{minipage}[c]{\textwidth}
\centering
\textit{make vdc}
\end{minipage}

The \vampire data converter, or \vdc, is run to produce the input files needed.
To generate the input files, an output type must be specified in the command line.
For example, to produce POV-Ray input files:

\noindent
\begin{minipage}[c]{\textwidth}
\centering
\textit{vdc -{}-povray}
\end{minipage}

Note that all command line parameters passed to vdc are not case sensitive.

\section*{Getting started}
\phantomsection\addcontentsline{toc}{section}{Getting started}

To generate the positions of your atoms, \textit{config:atoms} must be set in
the input file. This will produce .data and .meta files containing the atomic
and spin configuration of your system. Their frequency can be adjusted using the
\textit{config:atoms-output-rate} paramter or they can be set to be output at
the end of the simulation solely.

In addition, the format of the output can be text or binary format, the latter
can help with particulary large systems. Files written in binary format are
system specific and usually cannot be read by \vdc compiled on separate
hardware.

\subsection*{input}
{\footnotesize
\begin{verbatim}
#------------------------------------------
# data output
#------------------------------------------
config:atoms
config:output-nodes      = 12
config:atoms-output-rate = 1000
config:output-format     = binary
\end{verbatim}
}

\section*{Atomic visualization with rasmol}
\phantomsection\addcontentsline{toc}{section}{Atomic visualization with rasmol}

To visualise your system using Rasmol, simply run \vdc in the same directory as
your output with \textit{vdc -{}-xyz}. The \textit{config:atoms} files must be
present.

This produced a file called \textit{crystal.xyz}, which is a chemical file
format with information on the atomic positions. The format of the \textit{.xyz}
format is as follows:

\subsection*{.xyz}
{\footnotesize
\begin{verbatim}
<number of atoms>
comment line
<element> <X> <Y> <Z>
...
\end{verbatim}
}

The element in the .xyz file does not necessarily need to be the same as the
atoms used in your system. They can instead be chosen for a different colour
palette depending on the users requirements.

\section*{Atomic visualization with POV-Ray}
\phantomsection\addcontentsline{toc}{section}{Atomic visualization with POV-Ray}

To produce pictures of your material of punishable quality and high
configurability, it is also possible to use POV-Ray. After running \vdc, the
file "spins.pov" contains all the necessary information and an image may be
produced by using:

\noindent
\begin{minipage}[c]{\textwidth}
\centering
\textit{povray spins.pov}
\end{minipage}

When running povray it is also possible to select specific snapshots or ranges
to render using the following flags:

\noindent
\begin{minipage}[c]{\textwidth}
\centering
\textit{povray +KFF[initial frame number] +KFI[final frame number]}
\end{minipage}

For example, to render frame 9 only, you could use:

\noindent
\begin{minipage}[c]{\textwidth}
\centering
\textit{povray -W3600 -H2700 +A0.3 +KFI9 +KFF9 spins.pov}
\end{minipage}

\noindent where the "-W" and "-H" flags define the width and heigh of the image
(the resolution), and "+A" is used for antialiasing.

Output from \vdc can be customised in several ways, either by passing parameters
to the command line (using the '\textit{--}' notation), or by using a separate
\vdc input file. The \vdc input file is a plain text file containing parameters
and arguments which change the output behaviour. This should be helpful when
many parameters, or multiple vdc runs with the same parameters, are needed.
Comments can be included with the '\#' symbol and the special characters
' ,()\{\}[]:=!' can also be used but are ignored by \vdc.  By default this file
is called \textit{vdc-input}, and is read automatically. To change the name of
the \vdc input file, a command line parameter can be used:

\noindent
\begin{minipage}[c]{\textwidth}
\centering
\textit{vdc -{}-input-file [filename]}
\end{minipage}

There are many options that can be used to change all visualisation outputs
including Rasmol, Jmol and POV-Ray. To get help with the usage of these
parameters outside of the manual, it is also possible to print help messages
from the command line by using the \textit{-h} or \textit{-{}-help} command line
argument, followed by the name of a \textit{vdc-input} file parameter:

\noindent
\begin{minipage}[c]{\textwidth}
\centering
\textit{vdc -h [parameter-name]}
\end{minipage}

The help message should contain information on the parameter, as well as the
type of the associated value given, the default value and example usage of the
parameter.

\section*{General customisation options}
\phantomsection\addcontentsline{toc}{section}{General Customisation options}

It may be beneficial to use only smaller portions of your full system when
generating POV-Ray or Rasmol images. This can help with large systems where
rendering can become a time constraint, or systems made up of several elements
which might be less relevant for the visualisation. There are several similar
command line options which can be used to cut up the system in different ways:

{\zicf slice = float vector(6) [0-1 : default \{0,1,0,1,0,1\}]}
\phantomsection\addcontentsline{toc}{subsection}{slice} The first slice type
defines minimum and maximum values for each axis. Only atoms and spins inside
these boundaries are included in the visualisation. The parameters passed to
this argument are interpreted as fractional coordinates.

{\zicf slice-void = float vector(6) [0-1 : default \textit{not set}]}
\phantomsection\addcontentsline{toc}{subsection}{slice-void} This parameter will
\textit{remove} all atoms and spins inside the given borders. This can be used
to create cubic hollow systems where only surface atoms are shown,  removing a
very high percentage of atoms in the system, which can greatly reduce rendering
time for both POV-Ray and Rasmol.

{\zicf slice-sphere = float vector(3) [0-1 : \textit{xfrac},\textit{yfrac},\textit{zfrac}]}
\phantomsection\addcontentsline{toc}{subsection}{slice-sphere} The sphere slice
is also used to remove the atoms and spins at the centre of a system. This
particular parameter lends itself well to spherical systems as it removes a
spherical section of atoms. Three parameters are required, instead of six. Each
one defines a region, centred on the centre of the original system, along the
respective axis, equal to a fraction of the system size along that axis. As
these parameters are not necessarily equal to each other, this can be used to
create an ellipse of missing atoms at the centre of the system.

{\zicf slice-cylinder = float vector(4) [0-1 : \textit{xfrac},\textit{yfrac},\textit{zmin},\textit{zmax}]}
\phantomsection\addcontentsline{toc}{subsection}{slice-cylinder} This slice
parameter can be used to remove all atoms outside a cylindracal section by
defining the x,y-fractional sizes as well as a fractional minimum and maximum
along the z-axis.

{\zicf remove-materials = int [one or more values] }
\phantomsection\addcontentsline{toc}{subsection}{remove-materials} In some cases
whole materials are not relevant for visualisation purposes and can be
altogether removed. To use this command line parameter a list of material
indices need to be provided. Material indices start from 1.

\section*{POV-Ray Customisation options}
\phantomsection\addcontentsline{toc}{section}{POV-Ray Customisation options}

The following section contains a list of parameters that only affect POV-Ray
output configurations. If another output type is requested, these parameters
are ignored.

{\zicf frame-start = int [default 0]}
\phantomsection\addcontentsline{toc}{subsection}{frame-start} Depending on
output options used in \vampire, multiple frames may be rendered by \vdc.
\textit{frame-start} can be used to skip an initial number of frames.

{\zicf frame-final = int [default 0]}
\phantomsection\addcontentsline{toc}{subsection}{frame-final} Depending on
output options used in \vampire, multiple frames may be rendered by \vdc.
\textit{frame-final} can be used to skip later frames.

{\zicf camera-position = float vector(3) [(-1,1) : default \textit{not set}]}
\phantomsection\addcontentsline{toc}{subsection}{camera-position} POV-Ray camera
position, set using fractional coordinates. Camera distance from look at point
is calculated automatically however it can be changed by using
\textit{camera-zoom}.

{\zicf camera-look-at = float vector(3) [(-1,1) : default \textit{not set}]}
\phantomsection\addcontentsline{toc}{subsection}{camera-look-at} POV-Ray camera
look at position, set using fractional coordinates. The position is a location
in the bounding box of the system, with centre (0,0,0).

{\zicf camera-zoom = float vector(3) [0-$\infty$ : default \textit{not set}]}
\phantomsection\addcontentsline{toc}{subsection}{camera-zoom} The default
distance from the camera is automatically calculated according to the size of
the system. This can be increased or reduced using \textit{camera-zoom} to
multiply the default distance. Values less than 1.0 reduce the distance while
values above 1.0 increase it.

{\zicf background-colour = string [default Gray30]}
\phantomsection\addcontentsline{toc}{subsection}{background-colour} POV-Ray
includes various predefined colours such as \textit{White, Black, Gray}.
Misspelled colour names will not be detected by vdc but will cause error
in POV-Ray.

{\zicf atom-sizes = float [one or more : default 1.2]}
\phantomsection\addcontentsline{toc}{subsection}{atom-sizes} POV-Ray atom sizes.
Atoms are represented by spheres with a defined radius. Individual materials can
have different atoms sizes by including a list of floats, starting from
material 1.

{\zicf arrow-sizes = float [one or more : default 2.0]}
\phantomsection\addcontentsline{toc}{subsection}{arrow-sizes} POV-Ray arrow
sizes. Individual materials can have different arrow sizes by including a list
of floats, starting from material 1.

{\zicf povray-sticks = bool [true or false : default false]}
\phantomsection\addcontentsline{toc}{subsection}{arrow-sizes} Adds sticks
between nearby atoms for visualisation in POV-Ray to better visualise the
crystal structure. Note that the sticks algorithm is inefficient and only
suitable for a small number of atoms (a few thousand).

{\zicf sticks-cutoff = float [one or more : default 2.0]}
\phantomsection\addcontentsline{toc}{subsection}{arrow-sizes} Defines the
cutoff range for sticks, and should be approximately the nearest
neighbour distance between atoms.

\begin{figure*}[!h]
\center
\includegraphics[width=4cm]{figures/CBWR_colourmap.jpg}
\\ CBWR colourmap
\label{fig:CBWR_colourmap}
\end{figure*}

{\zicf colourmap = string [default CBWR]}
\phantomsection\addcontentsline{toc}{subsection}{colourmap} By default, a 1D
colourmap is used. Aligned along the z-axis, spins in the \{0,0,1\} direction
are red, while spins antiparallel to this \{0,0,-1\} are blue. Between these
values, the colour transitions to white around the xy-plane. This corresponds to
the \textit{CBWR} colourmap, a cyclic blue-white-red map, which lends itself
well to 1D or 2D spin sytems where there are two principle spin directions, such
as antiferromagnets and ferrimagnets. Some care must be taken to align the
principle spin directions with the z-axis, as this is the axis along which
colour is applied. This can also be changed using the \textit{vector-z} input
parameter. There are several choices of possible colourmap configurations, the
ones provided by default are made to be perceptually uniform and in some cases
take account of colourblindness. Information on the colourmaps, the importance
of perceptually uniform maps and how to adapt and use different maps can be
found from \textit{"Peter Kovesi. Good Colour Maps: How to Design Them. 2015"}.

\begin{figure*}[!h]
\center
\includegraphics[width=4cm]{figures/C2_colourmap.jpg}
\\ C2 colourmap
\label{fig:C2_colourmap}
\end{figure*}

The \textit{C2} coloumap is also cyclic and useful for 3D magnetic systems such
as vortex states. It has four principle directions of magenta, yellow, green and
blue. As it is cyclic, there will be a smooth transition between colour at all
angles, irrespective of what is chosen as the zero degree spin direction.
Systems which benefit from this colourmap may also use the \textit{3D} parameter
which applies a brightness effect along the x-axis.

\vspace{5pt}
\begin{figure*}[!h]
\center
\includegraphics[width=7cm]{figures/BWR_colourmap.jpg}
\\ BWR colourmap
\label{fig:BWR_colourmap}
\end{figure*}

The \textit{BWR} colourmap is very similar in properties to the CBWR map however
it is not cyclic. This mean that spins along the positive z-axis will be red
with a small positive y-component and blue with a small negative y-component.
There will be an immediate flip from bright red to blue as this transition
occurs. This can be used to emphasise the transition between spin directions.
The transition point can be changed by using \textit{vector-z}.

\vspace{5pt}
\begin{figure*}[!h]
\center
\includegraphics[width=7cm]{figures/Rainbow_colourmap.jpg}
\\ Rainbow colourmap
\label{fig:Rainbow_colourmap}
\end{figure*}

The \textit{Rainbow} colourmap can be used in 2D systems where spins are aligned
in many different directions such as high temperature simulations. While it is
still designed to be somewhat perceptually uniform, this is very difficult to do
with rainbow palettes hence its use typically loses detail when compared to
other maps, however it is also one of the most vibrant.

{\zicf custom-colourmap = filename }
\phantomsection\addcontentsline{toc}{subsection}{custom-colourmap} A user
defined colourmap can also be used. To apply a different map, a file containing
256 colours in the RBG format must be provided in the same directory that \vdc
is run. RGB values must be space separated, with no other information such as
line numbers. The beginning of an example colourmap is shown below.

Pregenerated perceptually uniform colourmaps of various forms, including those
included in vampire by default, can be found in
\textit{peterkovesi.com/projects/\newline colourmaps/index.html} under the
Download secion.

\subsection*{custom\_colourmap\_file}
{\footnotesize
\begin{verbatim}
0.000000 0.000000 0.000000
0.005561 0.005563 0.005563
0.011212 0.011219 0.011217
0.016877 0.016885 0.016883
0.022438 0.022448 0.022445
0.027998 0.028011 0.028008
0.033540 0.033554 0.033551
0.039316 0.039333 0.039329
0.044700 0.044719 0.044714
0.049695 0.049713 0.049709
0.054322 0.054343 0.054338
\end{verbatim}
}

{\zicf 3D = bool [default false]}
\phantomsection\addcontentsline{toc}{subsection}{3D} POV-Ray images produced by
\vdc can have a 3D brightening effect applied. Spins which do not line only in
the yz-plane have their brightness adjusted according to their x-axis spin
component.

{\zicf vector-z = float vector(3) [default \{0,0,1\}]}
\phantomsection\addcontentsline{toc}{subsection}{vector-z} The principle axis,
along which colour is applied, is the z-axis. This determines where colours will
occur depending on the colourmap being used. By default the \textit{CBWR} map is
used; spins along the positive z-direction are red, those along the negative
z-direction are blue, and spins aligned along the xy-plane are white.

In many cases, the overall magnetic moment does not necessarily lie along the
z-axis. To remedy this, a new \textit{vector-z} may be defined. To redefine the
z-axis, use the parameter \textit{vector-z} followed by a direction vector. This
does not need to be normalised.

For example, if the user defines \textit{vector-z} = \{1,1,1\}, spins along the
\{1,1,1\} direction will be red, \{-1,-1,-1\} will be blue and those
perpendicular to the given axis will be white. Brackets can be omitted.

{\zicf vector-x = float vector(3) [default \{0,0,1\}]}
\phantomsection\addcontentsline{toc}{subsection}{vector-x} In some cases, the
colourmap may not be symmetric along the default xy-plane, such as the
\textit{C2} colourmap. Here, spins along positive-y are magenta, while those
antiparallel are green. This can be adjusted using a similar command line
argument \textit{vector-x}, however this argument cannot be used without first
defining \textit{vector-z}.

{\zicf afm = int [one or more values] }
\phantomsection\addcontentsline{toc}{subsection}{afm} POV-Ray visualization of
antiferromagnets can be difficult due to the contrast of colours of antiparallel
spins. To remedy this, it is possible to define materials as antiferromagnetic.
These materials will have their colours flipped so that they match neighbouring
spins while their spin direction remains antiferromagnetic.

\section*{Micromagnetic visualization with POV-Ray}
\phantomsection\addcontentsline{toc}{section}{Micromagnetic visualization with POV-Ray}
cell2povray

%macros

%customization

%colouring options

\section*{Visualization Movies}
\phantomsection\addcontentsline{toc}{section}{Visualization Movies}
